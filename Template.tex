% PRÄAMBEL
% ***************************************************************************************************

\documentclass[smallheadings,headsepline,11pt,oneside,a4paper]{scrbook}

% Hier gibt man an, welche Art von Dokument man schreiben möchte.
% Möglichkeiten in {}: scrartcl, scrreprt, scrbook, aber auch: article, report, book

\usepackage[ngerman]{babel} 		% ermöglicht deutsche Silbentrennung und direkte Eingabe von Umlauten, ...
\usepackage[utf8]{inputenc} 	% teilt LaTeX die Texcodierung mit. Bei Windowssystemen: ansinew
\usepackage[T1]{fontenc} 		% ermöglicht die Silbentrennung von Wörtern mit Umlauten

\usepackage[usenames,dvipsnames]{color}

% PDF wird mit Lesezeichen (verlinktes Inhaltsverzeichnis) versehen (bei Betrachtung mit Acrobat Reader sichtbar)
\definecolor{farbelink}{rgb}{0,0,0}		%Definition der Linkfarbe für das PDF-File
\usepackage[pdftex,colorlinks=true,urlcolor=farbelink,linkcolor=farbelink,citecolor=farbelink]{hyperref}

\usepackage{lscape}

\usepackage{amssymb,amsmath}

\usepackage{geometry} 
\geometry{a4paper} 
\usepackage[parfill]{parskip}    								% Activate to begin paragraphs with an empty line
\usepackage[pdftex]{graphicx}
\usepackage{epstopdf}
\DeclareGraphicsRule{.tif}{png}{.png}{`convert #1 `dirname #1`/`basename #1 .tif`.png}
\usepackage[margin=10pt,font=small,labelfont=bf]{caption}		%Package für kleinere Schrift unter den Bildern

\usepackage[usenames]{color}

\usepackage{listings} 									%Einfügen von Programmcode
\lstset{numbers=left, numberstyle=\tiny, numbersep=5pt} 			%mit Zeilennummerierung links
\lstset{language=[77]Fortran}								%Programmiersprache: Fortran77


%text with 2.5cm on both sides (A4=210mm)
\setlength{\textwidth}{160mm}
\setlength{\evensidemargin}{0in}
\setlength{\oddsidemargin}{0in}
\setlength{\columnsep}{0.25in}
%
%text with 2.5cm on top and 2.5cm on bottom (A4=297mm)
\setlength{\textheight}{247mm}
\setlength{\topmargin}{0pt}
\setlength{\headsep}{25pt}
\setlength{\headheight}{0pt}


%\typearea{12} % Breite des bedruckten Bereiches vergrössern (funktioniert nur in \documentclass mit: scrreprt, scrartcl, scrbook)

\clubpenalty = 10000 		% schliesst Schusterjungen aus
\widowpenalty = 10000 		% schliesst Hurenkinder aus


\usepackage{fancyhdr}
\pagestyle{fancy}			%Selbst gestaltete Kopf-/Fusszeilen

%%% Fancy Header %%%%%%%%%%%%%%%%%%%%%%%%%%%%%%%%%%%%%%%%%%%%%%%%%%%%%%%%%%%%%%%%%%
% Fancy Header Style Options
\fancyhf{}
\fancyfoot{} 

\renewcommand{\chaptermark}[1]{         		% Lower Case Chapter marker style
\markboth{\chaptername\ \thechapter.\ #1}{}} 	%
\renewcommand{\sectionmark}[1]{         		% Lower case Section marker style
\markright{\thesection.\ #1}}         			%
\fancyfoot[R]{\thepage}    					% Page number in the right on every page
\fancyhead[L]{\leftmark}       				% Chapter in the right on even pages
\fancyhead[L]{\rightmark}     				% Section in the left on odd pages
\renewcommand{\headrulewidth}{0.3pt}   		% Width of head rule

\fancypagestyle{plain}{%
\fancyhf{} 								% clear all header and footer fields
\fancyfoot[R]{\thepage}    					% Page number in the right on every page
\fancyhead[L]{\leftmark}       				% Chapter in the right on even pages
\renewcommand{\headrulewidth}{0.3pt}}   	% Width of head rule


\setlength{\textheight}{21cm}				%Vergrössert die Höhe des bedruckbaren Bereichs


\begin{document}

\frontmatter							%römische Seitenzahlen

% TITELSEITE
% ***************************************************************************************************
% use FS or HS to identify the semester, FS (Fruehling) for spring semester, HS (Herbst) for fall semester

\begin{titlepage}
\begin{center}
    \vspace*{1cm}
    {\huge \bfseries  Geschwindigkeitsmessung am instationärem Strömungsfeld eines piezoelektrischen Lüfters mittels Particle Image Velocimetry\\ }
    \vspace{2cm}
    {\large 
	Alexander Held\\
	~\\
	Studiengang Maschineningenieurwissenschaften Bachelor\\
	\vspace{3.5cm}
	Bachelorarbeit FS 2016\\
	~\\
	Institut für Fluiddynamik\\
	ETH Zürich\\
    }

\vspace{\stretch{1}}



{\large
	Betreuer: Prof. T. Rösgen\\[\baselineskip]
	Professor: Prof. T. Rösgen
}
\end{center}

\vspace*{2cm} % a bit of space at the bottom of the page

\end{titlepage}

%\clearpage\null


% Einfügen einer leeren Seite nach dem Titelblatt
\begin{titlepage}
\thispagestyle{empty}
\newpage
\mbox{}
\end{titlepage}


% ABSTRACT
% ***************************************************************************************************
% ABSTRACT
% ***************************************************************************************************
%mit horizonztalen Linien
\begin{tabular}{p{0.9\textwidth}}
\chapter*{Abstract}

Text

\end{tabular}




% INHALTSVERZEICHNIS
% ***************************************************************************************************

\tableofcontents
% Dieser Befehl erstellt das Inhaltsverzeichnis. Damit die Seitenzahlen korrekt sind, muss das Dokument zweimal gesetzt werden!



% HAUPTTEIL
% ***************************************************************************************************
% Folgende Befehle stehen für die Gliederung zur Verfügung: \chapter \section \subsection \subsubsection \paragraph
% Für Anführungszeichen: "` (CH-Tastatur: ZUERST: Shift-Tast und Taste 2, DANN: Shift-Taste und Taste ^)
% Für Schlusszeichen: "' (CH- Tastatur: ZUERST: Shift-Tast und Taste 2, DANN:  '-Taste, rechts neben der Null)
% Für Neuen Abschnitt: Eine Zeile leer lassen.


\mainmatter	%arabische Seitenzahlen


%1. Einführung
%------------------------------------------------------------------------------------------------------------------------------
\chapter{Einführung}

%Kapitel enthält Einführung in das Thema und in die Arbeiten

%Weitere Kapitel

%2. Vorgehen
%------------------------------------------------------------------------------------------------------------------------------
\chapter{Vorgehensweise/Methode}
Um den Strömungsverlauf von dem Piezo Jet zu messen, wurde die "Particle Image Velocimetry" angewandt. Dazu wird dem Luftstrom kleine Rauchpartikel zugesetzt, welche mittels Lasersheet beleuchtet werden. Anschliessend wird eine CCD Kamera mit dem sog. Double Shutter Mode innerhalb kürzester Zeit zwei aufeinander folgende Bilder von dem Jet schiessen. Letzten Endes werden die Aufnahmen bearbeitet und mittels Cross Korrelation miteinander verglichen. Dadurch werden die Verschiebungen in Pixeln berechnet, mit der die Srömungscharakteristiken ermittelt werden können.
\section{Experimenteller Aufbau}
Für den Aufbau der Messanlage, wird eine CCD Karmera, ein Laser und eine Rauchquelle benötigt. Bei der verwendeten Karmera, handelt es sich um eine PixelFly CCD Scienticic Mode, welche an einem Balken befestigt wird und in der Höhe einstellbar ist.\\
 Der verwendete Pulslaser, ist der Minilite PIV von Continuum, welcher sich durch zwei seperate Pulslaser speziell für PIV Experiment eignet  und eine Wellenlänge von 532 nm verwendet. Der Laserstrahl wird mittels Spektroskop in eine Laseroptik gelenkt, wo dieser in ein Gauss verteiltes Lasersheet gewandelt wird. Die Optik wird mit einem Stativ weit über dem Lüfter positioniert, damit der Lichtschnitt paralell zum Piezo Luftspalt gerichtet ist.\\
 Der Lüfter ist mit dem Balken der Kamera fest auf einer Platte verschraubt, wodurch die Distanzen fix bleiben. Die Position des Lüfters ist so gerichtet, dass die Luft nach oben in Richtung Lichtschnitt geblasen wird. Direkt daneben wird ein kleiner Modell Rauchgenerator platziert, welcher durch Erhitzen einer Spule öl zum verdampfen bringt. Um zu verhindern, dass der Rauch zu schnell in die Umgebung diffundiert, wird ein Glaskasten über Lüfter und Rauchquelle gestellt. Dieser ist gross genug um die Strömung nicht zu beeinflussen und zusätzlich erreicht der Lichtschnitt noch den Lüfter.\\
 Für das Messen der elektrischen Ein- und Ausgangssignale am Mikrokontroller, wird ein Oszilloskop benutzt. Die Signale vom Lüfter, sowie an die Kamera und den Laser, werden mit einem Kontroller von LabSmith gesteuert, wodurch ein synchronisiertes Arbeiten ermöglicht wird.
 
\section{Theorie}
\subsection{Theoretischer Hintergrund}
Die Messungen am Strömungsfeld werden durch Particle Image Velocimetry durchgeführt, eine bekannte Messmethodik in der Fluiddynamik. Dafür werden Bilder von dem gestreuten Licht der Partikel im Laserschnitt gemacht, wodurch der Vorteil resultiert, dass die Strömung durch Messungen nicht beeinflusst wird. Um kleinste Auslenkungen der Partikel zu bekommen, müssen zwei Laserpulse kurz hintereinander gefeuert werden, welche zwei aufeinanderfolgende Bilder belichten. Die beiden Aufnahmen zeigen die Verschiebung der Partikel in Pixeln, nachdem sie miteinander korreliert wurden. Diese Korrelation wird mit einem Programm ausgewertet, welches die Bilder durch Kreuz Korrelation miteinander vergleicht.\\
Diese Technik wurde schon bei anderen "Piezo Fans" ausgenutzt, jedoch handelt es sich dabei um ein anderes Modell. Deshalb gibt es noch keine ähnlichen Methodiken der Strömungsmessung an dem "Piezo Jet", wie im vorliegendem Fall.
\subsection{Synchronisierung des Triggers}
Um zu Erreichen, dass zwei kurze Laserpulse mit der Kamera gekoppelt werden, ist es notwendig die beiden Pulssignale zum Feuern der Laser mit dem Trigger der CCD zu synchronisieren. Die Logik der Pulse ist in Abbildung dargestellt.\\
Dadurch, dass die CCD auf fps beschränkt ist darf der Abstand zweier Trigger minimal ms betragen. Mit dem Double Shutter Modus ermöglicht man die erste Belichtung des Bildes variabel einzustellen und kurz danach ein nächstes Bild zu schiessen, bei dem die Belichtungszeit auf die Auslesezeit des ersten Bildes beschränkt ist.\\
Um zwei kurz aufeinanderfolgende Bilder zu bekommen, wird erste Laserpuls am Ende der ersten Belichtungszeit gesetzt und der zweite am Anfang des zweiten Belichtungsrahmens. Die Technologie der CCD erlaubt es durch "Interline-Transfer-Sensoren" zwei Belichtungsphasen im kurzem Abstand zu bekommen. Diese Technik hat PIV erst möglich gemacht.\\
\subsection{Signalverarbeitung am Mikrokontroller}
Um ein sinnvolles Startsignal für den Trigger der Kamera zu bekommen, wurde als erste Variante der Input und Output am Kontroller gemessen. Da man ein bei einem vielfachen der Periode triggern sollte, um immer am gleichen Punkt im Schwinungszyklus zu sein, wurde mit Hilfe der Kontrollerbox von LabSmith das PWM Signal nach der Periode abgetastet. Mit dem Oszilloskop für die Signaldarstellung wurde die Periode nach der folgenden Formel gemessen.\\
Da die minimale Distanz der Trigger ms ist, müssen 12 Perioden des Sinussignals abgewartet werden, das bedeutet triggern nach der Formel mit N=12.\\
Mit zweite Variante wurde durch einen AC/DC Wandler ein manuell einstellbarer Sinus an die Piezo Platten geschickt. Dadurch wurde als Triggersignal die Impulssignale des Wandlers genommen. Hier war gab es nun ein Puls pro Sinus, wodurch jeder Nte Trigger gross genug gewählt wurde um die Auslesezeit der kamera zu berücksichtigen.   
\chapter{Ergebnisse}
\section{Validierung}
\section{Ensemble Averaging}
\section{Unmittelbarer Strömungszyklus}
\include{Zusammenfassung}
%Literaturverzeichnis
%------------------------------------------------------------------------------------------------------------------------------
\include{Literaturverzeichnis}

%Anhang
%------------------------------------------------------------------------------------------------------------------------------
\appendix
\renewcommand{\chaptermark}[1]{         		% Benutze "Anhang" statt "Kapitel" in den Kopfzeilen
\markboth{Anhang\ \thechapter.\ #1}{}} 

\include{Appendix}



%ABBILDUNGSVERZEICHNIS
% ***************************************************************************************************
%\clearpage
%\phantomsection
%\addcontentsline{toc}{chapter}{Abbildungsverzeichnis}
%\listoffigures

%TABELLENVERZEICHNIS
% ***************************************************************************************************
%\clearpage
%\phantomsection
%\addcontentsline{toc}{chapter}{Tabellenverzeichnis}
%\listoftables


\end{document}
